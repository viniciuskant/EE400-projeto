\section{Considerações Finais}

O projeto exemplifica a aplicação prática de conceitos de paralelismo para resolver um problema computacionalmente intensivo. A escolha da linguagem C permitiu um controle eficiente do processo, enquanto a opção de não utilizar CUDA manteve a simplicidade e portabilidade do projeto, respeitando as restrições de infraestrutura dos alunos da disciplina.

Além disso, os exercícios realizados foram fundamentais para determinar os intervalos de interesse e compreender o conceito de pontos fixos, permitindo uma análise mais precisa e fundamentada do problema abordado. Eles também permitiram observar a simetria vertical do problema, o que possibilitou uma otimização significativa ao processar apenas metade dos dados.
