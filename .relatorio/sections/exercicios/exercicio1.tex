\subsection*{Exercício 1: Exploração do Conjunto de Mandelbrot}

Considerando o conjunto de Mandelbrot como descrito na introdução, seja \( M \) o
conjunto de pontos que o compõem, \( f_c : \mathbb{C} \to \mathbb{C} \) a função geradora desses pontos, dada por \( f_c (z) = z^2 + c \), para todo
\( c \in \mathbb{C} \), de forma que \( |f_c^n (z)| < \infty \), quando \( n \to \infty \). Em outras palavras, a órbita de \( c \) não diverge a \( \infty \). No caso de
\( M \), temos a condição inicial \( z_0 = 0 \), ou seja, todas as órbitas partem do mesmo ponto.

\begin{enumerate}[label=(\alph*)]
    \item \textbf{Explicite os primeiros termos da órbita de \( f_c \) para um \( c \) fixo. Teste se os seguintes pontos estão em \( M \): \( c_1 = -2 \), \( c_2 = -2i \), \( c_3 = 0.35 e^{i\pi/4} \).}
    
        Para \( c_1 = -2 \), temos \( z_0 = -2 \), \( z_1 = 2 \), \( z_2 = 2 \), \( z_3 = 2 \), \( z_4 = 2 \), \( z_5 = 2 \), \( z_6 = 2 \), \( z_7 = 2 \), \( z_8 = 2 \) e \( z_9 = 2 \). Conclusão: A órbita não diverge, mas não está em \( M \) pois não satisfaz a condição de convergência.

        Para \( c_2 = -2i \), temos \( z_0 = -2i \) e \( z_1 = -4 - 2i \). Conclusão: A órbita diverge para \( \infty \). \( c_2 \notin M \).

        Para \( c_3 = 0.35 e^{i\pi/4} \) (aproximado como \( c_3 = 0.247487 + 0.247487i \)), temos \( z_0 = 0.247487 + 0.247487i \), \( z_1 = 0.247487 + 0.369987i \), \( z_2 = 0.171847 + 0.430622i \), \( z_3 = 0.091584 + 0.395489i \), \( z_4 = 0.099463 + 0.319928i \), \( z_5 = 0.155026 + 0.311129i \), \( z_6 = 0.174719 + 0.343954i \), \( z_7 = 0.159710 + 0.367678i \), \( z_8 = 0.137808 + 0.364931i \) e \( z_9 = 0.133304 + 0.348068i \). Conclusão: A órbita não diverge e parece convergir. \( c_3 \in M \).

    

    \item \textbf{Prove que \( M \) é simétrico em relação ao eixo \( x \).} \\

        Para provar que \( M \) é simétrico em relação ao eixo \( x \), considere um ponto \( c \in \mathbb{C} \) tal que \( c = a + bi \), onde \( a, b \in \mathbb{R} \). O conjugado complexo de \( c \) é dado por \( \overline{c} = a - bi \).

        Queremos provar, por indução matemática, que para toda \( n \in \mathbb{N} \),

        \[
        \overline{z_n} = w_n,
        \]

        onde:

        \[
        \begin{cases}
        z_0 = 0 \\
        z_{n+1} = f_c(z_n) = z_n^2 + c
        \end{cases}
        \quad \text{e} \quad
        \begin{cases}
        w_0 = 0 \\
        w_{n+1} = f_{\overline{c}}(w_n) = w_n^2 + \overline{c}.
        \end{cases}
        \]

        \textbf{Base da indução:} Para \( n = 0 \),

        \[
        \overline{z_0} = \overline{0} = 0 = w_0.
        \]

        Portanto, a propriedade vale para \( n = 0 \).

        \textbf{Hipótese:} Suponha que para algum \( n \geq 0 \),

        \[
        \overline{z_n} = w_n.
        \]

        \textbf{Passo:} Vamos mostrar que então

        \[
        \overline{z_{n+1}} = w_{n+1}.
        \]

        Calculando \( z_{n+1} \):

        \[
        z_{n+1} = f_c(z_n) = z_n^2 + c.
        \]

        Tomando o conjugado:

        \[
        \overline{z_{n+1}} = \overline{z_n^2 + c} = \overline{z_n}^2 + \overline{c}.
        \]

        Pela hipótese de indução, \( \overline{z_n} = w_n \), então:

        \[
        \overline{z_{n+1}} = w_n^2 + \overline{c} = f_{\overline{c}}(w_n) = w_{n+1}.
        \]

        \textbf{Conclusão:} Por indução matemática, a propriedade vale para todo \( n \in \mathbb{N} \):

        \[
        \overline{z_n} = w_n \quad \forall n \in \mathbb{N}.
        \]

        \bigskip

        \textbf{Implicação geométrica:} Isso significa que a sequência de iterações de \( f_c \), partindo de \( 0 \), tem comportamento simétrico em relação ao eixo real (eixo \( x \)) quando comparamos \( c \) e \( \overline{c} \).

    \item \textbf{O que acontece com a órbita de um ponto que está fora de \( M \)?} \\
    
        Se um ponto está fora de \( M \), sua órbita diverge para \( \infty \). Isso ocorre porque, por definição, os pontos em \( M \) são aqueles cujas órbitas permanecem limitadas. Para pontos fora de \( M \), a sequência \( \{z_n\} \) gerada por \( f_c \) cresce indefinidamente em magnitude, ou seja, \( |z_n| \to \infty \) quando \( n \to \infty \). Geometricamente, isso significa que esses pontos não pertencem ao conjunto de Mandelbrot.

    \item \textbf{O que acontece com a órbita de um ponto que está dentro de \( M \)?} \\

        Se um ponto está dentro de \( M \), sua órbita não diverge para \( \infty \). Em vez disso, ela permanece limitada e pode exibir diferentes comportamentos dinâmicos, dependendo do ponto \( c \):

        \begin{itemize}
            \item \textbf{Convergência a um ponto fixo ou periódico:} A órbita pode convergir para um ponto fixo ou para um ciclo periódico. Nesse caso, o ponto fixo ou o ciclo periódico é chamado de atrator.
            \item \textbf{Comportamento caótico:} Em algumas regiões de \( M \), a órbita pode exibir comportamento caótico, sem convergir para um ponto fixo ou ciclo periódico, mas ainda assim permanecendo limitada.
        \end{itemize}

        Geometricamente, isso significa que os pontos dentro de \( M \) estão associados a dinâmicas estáveis ou limitadas da função \( f_c \).

\end{enumerate}