\subsection{Exercício 2: Estudando \( f_c \) e sua dinâmica}

Pontos periódicos de uma função são aqueles cujos valores se repetem após um número finito de iterações, assim, existe um \( n \) (chamado de período) tal que \( f_c^n (\hat{z}) = \hat{z} \). Se \( n = 1 \), \( \hat{z} \) é chamado de ponto fixo. Observe que, para cada \( c \), os pontos periódicos de \( f_c \) serão diferentes dos pontos de sua órbita. Porém, se a órbita é convergente, existirá um subconjunto dela que converge a cada \( \hat{z} \). Nesse caso, ele é chamado de atrator e a seguinte desigualdade é válida: \( |(f_c^n)' (\hat{z})| < 1 \).

\begin{enumerate}[label=(\alph*)]
    \item \textbf{A função \( f_c^n \) é holomorfa? Se sim, qual sua derivada complexa?} \\
    
         Sim, a função \( f_c^n \) é holomorfa para todo \( n \in \mathbb{N} \). Como \( f_c(z) = z^2 + c \) é um polinômio, ela é holomorfa em todo o plano complexo \( \mathbb{C} \). Além disso, a composição de funções holomorfas também é holomorfa, logo a função iterada \( f_c^n(z) = \underbrace{f_c \circ f_c \circ \cdots \circ f_c}_{n \text{ vezes}}(z) \) é holomorfa.

        A derivada de \( f_c^n \) é obtida aplicando a regra da cadeia sucessivas vezes. Assim, temos:

        \[
        (f_c^n)'(z) = f_c'(f_c^{n-1}(z)) \cdot f_c'(f_c^{n-2}(z)) \cdots f_c'(z) = \prod_{k=0}^{n-1} f_c'(f_c^k(z))
        \]

        Como \( f_c(z) = z^2 + c \), sua derivada é \( f_c'(z) = 2z \). Substituindo:

        \[
        (f_c^n)'(z) = \prod_{k=0}^{n-1} 2 f_c^k(z) = 2^n \cdot \prod_{k=0}^{n-1} f_c^k(z)
        \]

        Portanto, \( f_c^n \) é holomorfa e sua derivada é dada pelo produto acima.


    \item \textbf{Ache a região de \( M \) que contém os pontos \( c \) de forma que \( f_c \) possua um ponto fixo. Faça o mesmo para os \( c \) que fazem \( f_c \) ter um ponto de período igual a 2. Identifique essas duas regiões em um mesmo gráfico}. \\
    
        \[
        \begin{array}{|c|c|c|}
        \hline
        \textbf{Iteração} & \textbf{Expressão} & \textbf{Função} \\
        \hline
        1 & z_n & f_c^{(1)}(z_n) = z_n^2 + c \\
        2 & z_{n+1} = z_n^2 + c & f_c^{(2)}(z_n) = (z_n^2 + c)^2 + c \\
        \hline
        \end{array}
        \]


        \begin{itemize}
            \item Para ponto fixo:
                Agora igualando para os valores de \(z_n\) e \(f^{1}_c\), obtemos os pontos fixos:
            
                        
                    \begin{equation}
                        z_n =  z_n^2 + c
                    \label{eq:questao_2_item_b_1}
                    \end{equation}
           
                Resolvendo a equação quadrática:

                \[
                z^2 + c = z \implies z^2 - z + c = 0
                \]

                As raízes da equação são dadas por:

                \[
                z = \frac{1 \pm \sqrt{1 - 4c}}{2}
                \]

                Para que o ponto fixo seja atrator, a condição \( |f_c'(z)| < 1 \) deve ser satisfeita. Como \( f_c'(z) = 2z \), temos:

                \[
                |2z| < 1 \implies |z| < \frac{1}{2}.
                \]

                Substituindo \( z = \frac{1 \pm \sqrt{1 - 4c}}{2} \):

                \[
                \left| \frac{1 \pm \sqrt{1 - 4c}}{2} \right| < \frac{1}{2} \implies |1 \pm \sqrt{1 - 4c}| < 1.
                \]

                Portanto, a região de \( c \) que satisfaz essa condição pode ser determinada resolvendo a desigualdade acima. Para identificar essa região, foi implementado um código que gera o gráfico correspondente. Para executar o código, utilize o seguinte comando:

                \[
                \texttt{bin/mandelbrot --questoes}
                \]

                \begin{figure}[H]
                    \centering
                    \subfloat[Regiões de \( M \) contendo os pontos \( c \) para os quais \( f_c \) possui um ponto fixo.]
                    {
                        \includegraphics[width=0.3\textwidth]{figures/questao_2_parte1.png}
                        \label{fig:mandelbrot_regions_fixed_points}
                    }
                    \hfill
                    \subfloat[Regiões de \( M \) clássica.]
                    {
                        \includegraphics[width=0.3\textwidth]{figures/questao2_mandelbrot.png}
                        \label{fig:mandelbrot_regions_period_2}
                    }
                    \hfill
                    \subfloat[Sobreposição das regiões de \( M \) contendo os pontos \( c \) para os quais \( f_c \) possui um ponto fixo.]
                    {
                        \includegraphics[width=0.3\textwidth]{figures/soobreposicao_questao_2_parte1.png}
                        \label{fig:mandelbrot_regions_combined_overlay}
                    }
                \end{figure}




            \item  Para período igual a 2:
                Agora igualando para os valores de \(z_n\) e \(f^{2}_c\), obtemos os pontos de período 2 (\( n = 2 \)):
                                
                \begin{equation}
                    (z^2 + c)^2 + c - z = 0
                    \label{eq:parte2}                    
                \end{equation}

                Para cada solução \( z \), verificamos se \( f_c(z) \neq z \) (ou seja, não é ponto fixo) e se é atrator, satisfazendo a condição \( |4z(z^2 + c)| < 1 \).

                A condição para atrator de período 2 é:

                \begin{equation}
                    |4z(z^2 + c)| < 1
                    \label{eq:atrator}                    
                \end{equation}

                Para encontrar a região dos \( c \) em que \( f_c \) tem um ponto de período 2 atrator, resolvemos numericamente a equação acima. O método consiste em discretizar o plano complexo em uma grade de pontos e, para cada ponto \( c \), verificar se existe algum \( z \) que satisfaz \autoref{eq:atrator}. Para isso, iteramos sobre valores iniciais de \( z \) e calculamos a função \( f_c \) até encontrar uma solução que satisfaz as condições de atrator. Para esse pontos verificamos se satisfaz a \autoref{eq:parte2} com uma tolerância de 0.01. Após o cálculo, o resultado é salvo como uma imagem que mostra as regiões de \( M \) contendo os pontos \( c \) para os quais \( f_c \) possui um ponto de período igual a 2.

                A figura abaixo mostra as regiões de \( M \) contendo os pontos \( c \) para os quais \( f_c \) possui um ponto fixo e um ponto de período igual a 2.

                
                \begin{figure}[H]
                    \centering
                    \subfloat[Regiões de \( M \) contendo os pontos \( c \) para os quais \( f_c \) possui um ponto de período igual a 2.]
                    {
                        \includegraphics[width=0.3\textwidth]{figures/questao_2_parte2.png}
                        \label{fig:mandelbrot_regions_period_2}
                    }
                    \hfill
                    \subfloat[Regiões de \( M \) clássica.]
                    {
                        \includegraphics[width=0.3\textwidth]{figures/questao2_mandelbrot.png}
                        \label{fig:mandelbrot_regions_classic}
                    }
                    \hfill
                    \subfloat[Sobreposição das regiões de \( M \) contendo os pontos \( c \) para os quais \( f_c \) possui um ponto fixo e um ponto de período igual a 2.]
                    {
                        \includegraphics[width=0.3\textwidth]{figures/soobreposicao_questao_2_parte2.png}
                        \label{fig:mandelbrot_regions_combined_overlay}
                    }
                \end{figure}

        \end{itemize}




    \item \textbf{Prove que um ponto \( c \) pertence a \( M \) se \( |z_n| \leq 2 \), para todo \( n = 1, 2, \ldots \). }\\

        Suponha, por contraposição, que \( |z_n| > 2 \) para algum \( n \geq 1 \). Vamos mostrar que, nesse caso, a sequência \( \{z_n\} \) diverge, ou seja, \( |z_n| \to \infty \) quando \( n \to \infty \).

        Sabemos que \( f_c(z) = z^2 + c \). Assim, para \( |z_n| > 2 \), temos:

        \[
        |z_{n+1}| = |f_c(z_n)| = |z_n^2 + c| \geq |z_n|^2 - |c|.
        \]

        Como \( |z_n| > 2 \), temos \( |z_n|^2 > 4 \). Além disso, \( |c| \) é finito, então:

        \[
        |z_{n+1}| > |z_n|^2 - |c| > 4 - |c|.
        \]

        Portanto, \( |z_{n+1}| > |z_n| \) para \( |z_n| > 2 \), o que implica que a sequência \( \{|z_n|\} \) é estritamente crescente para \( |z_n| > 2 \). Como \( |z_n| \) cresce indefinidamente, concluímos que \( |z_n| \to \infty \) quando \( n \to \infty \).

        Por contraposição, se \( |z_n| \leq 2 \) para todo \( n \geq 1 \), então a sequência \( \{z_n\} \) não diverge, ou seja, \( c \in M \).


    \item \textbf{Prove que o intervalo de números reais puros que pertencem a \( M \) é \( [-2, 0.25] \). }\\
    
        Para provar que o intervalo de números reais puros que pertencem a \( M \) é \( [-2, 0.25] \), iremos fazer o limite de \( n \) indo a \( \infty \) e analisar o comportamento da função \( f_c(z) = z^2 + c \).

        Para isso, consideramos a condição de que o módulo da derivada da função iterada deve ser menor que 1 para garantir que a órbita permaneça limitada:

        \[
        \lim_{n \to \infty} \left| \frac{z_{n+1}}{z_n} \right| < 1
        \]

        Como o limite \(n \to \infty\) está no índice do termo, podemos fazer uma substituição para deixar a notação menos carregada, substituiremos \(z_n\) por \(x\).

        \[
        \left| \frac{x^2 + c}x \right| < 1
        \]

        \[
        \left| x^2 + c \right| < \left| z_n \right| 
        \]

        \[
        \left| x^2  -x + c \right| < 0
        \]

        Podemos analisar o comportamento da função quadrática \( {z_n}^2 - z_n + c \). Para que essa função tenha raízes reais, o discriminante deve ser não negativo:

        \[
        \Delta = (-1)^2 - 4 \cdot 1 \cdot
        c = 1 - 4c \geq 0
        \]
        Resolvendo a desigualdade:
        \[
        1 - 4c \geq 0 \implies c \leq 0.25
        \]


        Pelo resultado do item anterior, sabemos que \( c \) deve ser maior ou igual a \( -2 \) para que a órbita não diverja. Assim, temos:

        \[
        -2 \leq c \leq 0.25
        \]
    \end{enumerate}